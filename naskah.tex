\documentclass[12pt,a4paper]{report}

\usepackage[utf8]{inputenc}
\usepackage{graphicx}
\usepackage{geometry}
\usepackage{setspace}
\usepackage{times}
\usepackage{tocloft}
\usepackage{caption}
\usepackage{fancyhdr}
\usepackage{indentfirst}
\usepackage{titlesec}
\usepackage{enumitem}
\usepackage[hang,flushmargin]{footmisc}
\usepackage{float}
\usepackage{hyperref}  % Untuk \url
\usepackage{etoolbox}
\usepackage{array}

\makeatletter
\def\@biblabel#1{}
\renewcommand\@cite[2]{{#1\if@tempswa,\nolinebreak[3] #2\fi}}
\makeatother

\titleformat{\chapter}[display]
  {\normalfont\LARGE\bfseries\filcenter}  % <-- centered and bold
  {}                                      % No chapter number
  {0pt}
  {\LARGE}

% Fix spacing before unnumbered chapters (like bibliography)
\titlespacing*{\chapter}{0pt}{-1cm}{2em}  % Adjust -2.5cm to control top margin

% Macro untuk judul bab manual (BAB II dst.) agar rapi dan konsisten
\newcommand{\customchapter}[2]{%
    \chapter*{}%
    \addcontentsline{toc}{chapter}{#1 #2}%
    \setcounter{section}{0}%
    \renewcommand{\thesection}{\thechapter.\arabic{section}}%
    \vspace*{-3.5cm}
    \begin{center}
        {\LARGE \textbf{#1}} \\[1em]
        {\LARGE \textbf{#2}}
    \end{center}
}


\captionsetup[figure]{labelfont=bf, name=Gambar, labelsep=period, font=normalsize}
\captionsetup[table]{labelfont=bf, name=Tabel, labelsep=period, font=normalsize}


\renewcommand{\footnotesize}{\fontsize{10pt}{5pt}\selectfont}
\newcommand{\sumbergambar}[1]{%
  \begingroup
  \fontsize{9pt}{9.5pt}\selectfont  % lebih kecil dari footnotesize
  \setstretch{1}
  (#1)
  \endgroup
}


\titlespacing{\section}{0pt}{0em}{0em}
\titlespacing{\subsection}{0pt}{0em}{0em}
\titlespacing{\subsubsection}{0pt}{0em}{0em}
\titleformat{\section}[hang]{\normalfont\bfseries}{\thesection}{0.2em}{}
\titleformat{\subsection}[hang]{\normalfont\bfseries}{\thesubsection}{0.2em}{}
\titleformat{\subsubsection}[hang]{\normalfont\normalsize\bfseries}{\thesubsubsection}{0.2em}{}

% MARGIN sesuai standar skripsi
\geometry{top=4cm, bottom=3cm, left=4cm, right=3cm}

% SPACING 1.5 spasi
\onehalfspacing

% INDENTASI paragraf
\setlength{\parindent}{1.5em}
\setlength{\parskip}{0pt}  % tidak ada spasi antar paragraf

% Format daftar isi, gambar, tabel
\setlength{\cftbeforetoctitleskip}{-0.5em}
\setlength{\cftaftertoctitleskip}{0pt}
\setlength{\cftbeforesecskip}{0pt}
\setlength{\cftbeforefigskip}{0pt}
\setlength{\cftbeforetabskip}{0pt}

\setlength{\cftafterloftitleskip}{-4em}  % Untuk daftar gambar
\setlength{\cftafterlottitleskip}{-4em}  % Untuk daftar tabel

% HEADER/FOOTER
\fancypagestyle{plain}{
    \fancyhf{}
    \fancyfoot[C]{\thepage}  % nomor halaman tengah bawah untuk halaman romawi
    \renewcommand{\headrulewidth}{0pt} % ← ini menghilangkan garis di header
}

\fancypagestyle{content}{
    \fancyhf{}
    \fancyhead[R]{\thepage}  % nomor halaman kanan atas untuk isi
    \renewcommand{\headrulewidth}{0pt}
}

\begin{document}

% =========================
% COVER
% =========================
\pagenumbering{gobble}
\thispagestyle{empty}

\begin{center}

    \vspace*{1cm}
    {\LARGE \textbf{KECERDASAN BUATAN DENGAN ALGORITMA GENETIKA DAN TRANSFORMER}}

    \vspace*{\fill}

    % Placeholder gambar/logo kampus
    \rule{0.3\textwidth}{0.2\textwidth} \\[1.5cm]

    {\large I Wayan Pio Pratama} \\[0.3cm]
    {\large NIM: 1208605000}

    \vspace*{\fill}

    {\large Fakultas Teknik Informatika} \\
    {\large Universitas Bali Scientist} \\
    {\large 2025}

\end{center}

\clearpage

% =========================
% HALAMAN AWAL: ROMAWI
% =========================
\pagenumbering{roman}
\setcounter{page}{2}
\pagestyle{plain}

\thispagestyle{plain}

\begin{center}
    \textbf{\LARGE LEMBAR PERSETUJUAN PEMBIMBING} \\[1.2cm]
\end{center}

\vspace{-0.4cm}
Yang bertanda tangan di bawah ini menerangkan bahwa Mahasiswa(i) sebagai berikut: \\[0cm]

\begin{tabular}{@{}lp{11cm}}
Nama & : I Wayan Pio Pratama \\
NIM & : 3201017002 \\
Program Studi & : S-1 Teknik Informatika \\
Judul & : \parbox[t]{11cm}{
  Kecerdasan Buatan Dengan Algoritma Genetika dan Transformer
}
\end{tabular}

\vspace{0.8cm}

Proposal ini telah dikonsultasikan dan disetujui oleh pembimbing untuk melaksanakan Ujian Proposal pada Semester Genap Tahun Akademik 2024/2025.

\begin{center}
    \textbf{Menyetujui} \\[0.5cm]
\end{center}

\begin{center}
    \begin{tabular}{p{0.48\textwidth} p{0.48\textwidth}}
    \begin{center}
    \textbf{Pembimbing I} \\[1.5cm]
    \textbf{I Wayan Pio Pratama} \\
    NIP. 19800901 200501 1 000
    \end{center}
    &
    \begin{center}
    \textbf{Pembimbing II} \\[1.5cm]
    \textbf{I Wayan Pio Pratama} \\
    NIP. 19800901 200501 1 000
    \end{center}
    \end{tabular}
\end{center}

\vspace{1cm}

\begin{center}
    \textbf{Mengetahui} \\[0.3cm]
    Ka. Program Studi S-1 Teknik Informatika \\
    Universitas Bali Scientist \\[1.5cm]
    \textbf{I Wayan Pio Pratama} \\
    NIP. 19800901 200501 1 000
\end{center}

\clearpage

\thispagestyle{plain}
\begin{small} % Gunakan ukuran font kecil untuk halaman ini

\begin{center}
    \textbf{\LARGE LEMBAR PERSETUJUAN DAN PENGESAHAN TUGAS AKHIR} \\[1em]
\end{center}

\noindent
Tugas Akhir oleh: I Wayan Pio Pratama NIM 12086050xx ini telah dipertahankan di hadapan dewan penguji pada tanggal 30 November 2025. \\[0.4cm]

\begin{tabbing}
\hspace{4cm} \= \kill
Judul Tugas Akhir \> : Kecerdasan Buatan dengan Algoritma Genetika dan Transformer \\
Nama \> : I Wayan Pio Pratama \\
NIM \> : 12086050xx
\end{tabbing}

\vspace{0.5cm}

\begin{center}
\textbf{Dewan Penguji}
\end{center}

\begin{center}
    \begin{tabular}{p{0.48\textwidth} >{\raggedleft\arraybackslash}p{0.48\textwidth}}
    \textbf{Ketua,} & \textbf{Anggota,} \\[1cm]

    \textbf{I Wayan Pio Pratama} & \textbf{I Wayan Pio Pratama} \\
    NIP. 19800901 200501 1 000 & NIP. 19800901 200501 1 000 \\[0.6cm]

    \textbf{Anggota,} & \textbf{Anggota,} \\[1cm]

    \textbf{I Wayan Pio Pratama} & \textbf{I Wayan Pio Pratama} \\
    NIP. 19800901 200501 1 000 & NIP. 19800901 200501 1 000
    \end{tabular}
\end{center}

\begin{center}
\textbf{Mengetahui}
\end{center}

\begin{center}
    \begin{tabular}{p{0.48\textwidth} p{0.48\textwidth}}
    \textbf{Ketua Program Studi,} & \makebox[\linewidth][r]{\textbf{Dekan Fakultas Teknik Informatika,}} \\[1cm]

    \textbf{I Wayan Pio Pratama} & \makebox[\linewidth][r]{\textbf{I Wayan Pio Pratama}} \\
    NIP. 19800901 200501 1 000 & \makebox[\linewidth][r]{NIP. 19800901 200501 1 000}
    \end{tabular}
\end{center}


\begin{center}
    \begin{tabular}{p{0.48\textwidth} p{0.48\textwidth}}
    \textbf{Tanggal Lulus}: 18 Januari 2026 & \raggedleft \textbf{Tanggal Wisuda}: .....................................
    \end{tabular}
\end{center}

\end{small}

\clearpage

\begin{center}
    \textbf{\LARGE KATA PENGANTAR}
\end{center}

Puji syukur penulis panjatkan ke hadirat Tuhan Yang Maha Esa, karena atas rahmat, karunia, dan hidayah-Nya, penulis dapat menyelesaikan Tugas Akhir yang berjudul
\textit{"Kecerdasan Buatan dengan Algoritma Genetika dan Transformer"}

Tugas Akhir ini disusun dalam rangka memenuhi salah satu syarat untuk memperoleh gelar Sarjana Teknik pada Program Studi Teknik Informatika, Fakultas Teknik Informatika, Universitas XYZ.

Penulis menyadari bahwa penyusunan Tugas Akhir ini tidak lepas dari bantuan, dukungan, dan doa dari berbagai pihak. Oleh karena itu, pada kesempatan ini penulis menyampaikan ucapan terima kasih yang sebesar-besarnya kepada:

\begin{enumerate}
    \item Bapak/Ibu Dosen Penguji yang telah memberikan masukan dan bimbingan selama proses sidang Tugas Akhir.
    \item Bapak/Ibu Dosen Pembimbing yang telah meluangkan waktu, tenaga, dan pikiran dalam memberikan arahan selama proses penyusunan.
    \item Kedua orang tua tercinta yang selalu memberikan doa, dukungan moral maupun materiil tanpa henti.
    \item Rekan-rekan seperjuangan yang telah memberi semangat, bantuan, dan kebersamaan selama masa studi dan penyusunan Tugas Akhir ini.
\end{enumerate}

Penulis menyadari bahwa Tugas Akhir ini masih jauh dari sempurna. Oleh karena itu, saran dan kritik yang membangun sangat diharapkan demi perbaikan di masa yang akan datang.

\vspace{1cm}

\begin{flushright}
Denpasar, Januari 2025\\[2cm] % [2cm] for signature space
Penulis
\end{flushright}

\clearpage

% =========================
% DAFTAR ISI
% =========================
\begin{center}
    {\Large \textbf{DAFTAR ISI}}
\end{center}
\renewcommand{\contentsname}{}
\tableofcontents
\clearpage

% =========================
% DAFTAR GAMBAR
% =========================
\begin{center}
    {\Large \textbf{DAFTAR GAMBAR}}
\end{center}
\renewcommand{\listfigurename}{}
\listoffigures
\clearpage

% =========================
% DAFTAR TABEL
% =========================
\begin{center}
    {\Large \textbf{DAFTAR TABEL}}
\end{center}
\renewcommand{\listtablename}{}
\listoftables
\clearpage

% =========================
% HALAMAN ISI: ARABIC
% =========================
\pagenumbering{arabic}
\setcounter{page}{1}
\pagestyle{content}

% =========================
% BAB I
% =========================
\setcounter{chapter}{1}
\customchapter{BAB I}{PENDAHULUAN}
\thispagestyle{plain}
\sloppy

\addcontentsline{toc}{chapter}{BAB I PENDAHULUAN}
\setcounter{section}{0}
\renewcommand{\thesection}{1.\arabic{section}}

\section{Latar Belakang}

Perkembangan teknologi kecerdasan buatan (Artificial Intelligence/AI) mengalami kemajuan pesat dalam beberapa dekade terakhir. AI telah diterapkan dalam berbagai bidang seperti pengolahan bahasa alami, pengenalan pola, optimisasi, dan pengambilan keputusan. Dua pendekatan yang semakin menonjol dalam pengembangan AI modern adalah algoritma genetika dan arsitektur transformer.

Algoritma genetika (AG) merupakan teknik pencarian dan optimisasi berbasis prinsip evolusi biologis seperti seleksi alam, mutasi, dan rekombinasi \cite{goldberg1989}\footnote{AG termasuk dalam keluarga algoritma evolusioner yang meniru proses evolusi alam untuk menemukan solusi optimal.}. Teknik ini telah terbukti efektif dalam menyelesaikan berbagai masalah komputasi kompleks yang sulit diselesaikan dengan pendekatan konvensional, terutama dalam pencarian solusi global pada ruang solusi yang luas dan tidak terstruktur.

Di sisi lain, transformer adalah arsitektur model deep learning yang diperkenalkan oleh Vaswani et al.\ \cite{vaswani2017}\footnote{Arsitektur ini menjadi fondasi model NLP besar seperti BERT dan GPT, yang secara signifikan meningkatkan akurasi dan efisiensi dalam berbagai tugas pemrosesan bahasa.} dan merevolusi bidang pemrosesan bahasa alami. Transformer mengandalkan mekanisme self-attention yang memungkinkan model memahami konteks data secara lebih efisien tanpa ketergantungan sekuensial seperti pada RNN atau LSTM.

Menggabungkan algoritma genetika dan transformer dalam satu kerangka kerja menawarkan potensi besar. AG dapat digunakan untuk mengoptimalkan parameter atau arsitektur model transformer secara evolusioner, sehingga menghasilkan model yang lebih efisien dan adaptif terhadap permasalahan tertentu. Dengan demikian, penelitian ini menjadi relevan dalam mendorong efisiensi dan efektivitas sistem AI masa kini.

\section{Rumusan Masalah}

Berdasarkan latar belakang tersebut, rumusan masalah dalam penelitian ini adalah sebagai berikut:
\begin{enumerate}[label=\arabic*., 
    leftmargin=3em, 
    labelwidth=1.5em, 
    labelsep=0em, 
    itemindent=0pt, 
    listparindent=1.5em, 
    align=left, 
    itemsep=0em,
    topsep=0pt, 
    partopsep=0pt]
    \item Bagaimana algoritma genetika dapat digunakan untuk mengoptimalkan performa model berbasis transformer?
    \item Apa saja parameter atau struktur dalam model transformer yang paling efektif untuk dioptimalkan menggunakan algoritma genetika?
    \item Sejauh mana kombinasi antara algoritma genetika dan arsitektur transformer meningkatkan performa dalam suatu studi kasus tertentu?
\end{enumerate}

\section{Tujuan Penelitian}

Penelitian ini bertujuan untuk:
\begin{enumerate}[label=\arabic*., 
    leftmargin=3em, 
    labelwidth=1.5em, 
    labelsep=0em, 
    itemindent=0pt, 
    listparindent=1.5em, 
    align=left, 
    itemsep=0em,
    topsep=0pt, 
    partopsep=0pt]
    \item Menganalisis peran algoritma genetika dalam optimisasi model transformer.
    \item Mengembangkan dan mengimplementasikan pendekatan hybrid antara algoritma genetika dan transformer.
    \item Mengevaluasi performa model yang dihasilkan dari kombinasi kedua metode tersebut pada studi kasus tertentu.
\end{enumerate}

\section{Manfaat Penelitian}

Adapun manfaat yang diharapkan dari penelitian ini, antara lain:

\textbf{Manfaat Teoritis}
\begin{enumerate}[label=\arabic*., 
    leftmargin=5em, 
    labelwidth=1.5em, 
    labelsep=0.5em, 
    itemindent=0pt, 
    listparindent=0pt, 
    align=left, 
    itemsep=0em,
    topsep=0pt, 
    partopsep=0pt]
    \item Menambah khazanah ilmu pengetahuan dalam bidang kecerdasan buatan, khususnya pada penggabungan metode optimisasi evolusioner dan arsitektur deep learning.
    \item Memberikan kontribusi dalam pengembangan pendekatan hybrid AI yang adaptif dan efisien.
\end{enumerate}

\textbf{Manfaat Praktis}
\begin{enumerate}[label=\arabic*., 
    leftmargin=5em, 
    labelwidth=1.5em, 
    labelsep=0.5em, 
    itemindent=0pt, 
    listparindent=0pt, 
    align=left, 
    itemsep=0em,
    topsep=0pt, 
    partopsep=0pt]
    \item Menyediakan solusi alternatif bagi pengembang sistem AI untuk mengoptimalkan model transformer.
    \item Dapat diterapkan pada berbagai permasalahan nyata seperti klasifikasi teks, prediksi, dan optimasi tugas-tugas berbasis data besar.
\end{enumerate}

\clearpage

% Tambahkan bab berikutnya:
\setcounter{chapter}{2}
\customchapter{BAB II}{TINJAUAN PUSTAKA}
\thispagestyle{plain}
\sloppy


\section{Kecerdasan Buatan}

Kecerdasan Buatan (Artificial Intelligence/AI) merupakan bidang dalam ilmu komputer yang berfokus pada pengembangan sistem yang mampu melakukan tugas-tugas yang membutuhkan kecerdasan manusia, seperti pembelajaran, penalaran, dan pengambilan keputusan. Teknologi AI telah diterapkan dalam berbagai bidang, seperti pengenalan suara, pengolahan bahasa alami, sistem rekomendasi, hingga kendaraan otonom.

\section{Algoritma Genetika}

Algoritma genetika (AG) adalah algoritma evolusioner yang menggunakan prinsip seleksi alam untuk mencari solusi optimal dalam ruang pencarian yang kompleks. AG dapat digunakan dalam berbagai bidang optimasi, baik numerik maupun kombinatorik.

\begin{figure}[H]
    \centering
    \includegraphics[width=0.6\textwidth]{gambar/genetika.png}
    \caption{Ilustrasi mekanisme dasar Algoritma Genetika}
    \label{fig:genetika}
    \vspace{0em}
    \sumbergambar{Sumber: Yang, J. dan Xing, C. (2019). \textit{Data Source Selection Based on an Improved Greedy Genetic Algorithm}. \textit{Symmetry}, 11(2), 273. Diakses dari: \url{https://doi.org/10.3390/sym11020273}}
\end{figure}

\vspace{-1em}
Pada Gambar \ref{fig:genetika}, dapat dilihat proses seleksi, crossover, dan mutasi yang membentuk dasar dari algoritma ini. Mekanisme tersebut diulang-ulang untuk mendapatkan individu terbaik yang memenuhi kriteria fitness tertentu.

\section{Arsitektur Transformer}

Transformer merupakan arsitektur deep learning yang diperkenalkan oleh Vaswani et al. (2017)\footnote{Vaswani, A., Shazeer, N., Parmar, N., Uszkoreit, J., Jones, L., Gomez, A. N., ... \& Polosukhin, I. (2017). \textit{Attention is All You Need}. Advances in Neural Information Processing Systems, 30.}, yang mengandalkan mekanisme \textit{self-attention} untuk memproses data sekuensial. Tidak seperti RNN dan LSTM yang bergantung pada urutan data secara bertahap, Transformer mampu memproses seluruh sekuens secara paralel, sehingga jauh lebih efisien.

\begin{table}[H]
    \centering
    \caption{Perbandingan Arsitektur RNN, LSTM, dan Transformer}
    \label{tab:perbandingan}
    \begin{tabular}{|c|c|c|c|}
        \hline
        \textbf{Aspek} & \textbf{RNN} & \textbf{LSTM} & \textbf{Transformer} \\
        \hline
        Paralelisme & Rendah & Rendah & Tinggi \\
        \hline
        Urutan Data & Bergantung & Bergantung & Independen \\
        \hline
        Efisiensi & Cukup & Baik & Sangat Baik \\
        \hline
    \end{tabular}
\end{table}

Dari Tabel \ref{tab:perbandingan}, terlihat bahwa Transformer memiliki keunggulan dalam hal efisiensi dan kemampuan memproses data secara paralel, menjadikannya pilihan utama dalam berbagai aplikasi Natural Language Processing (NLP) modern.

\subsection{Komponen Utama dalam Arsitektur Transformer}

Transformer terdiri dari dua blok utama, yaitu encoder dan decoder. Setiap blok terdiri dari beberapa layer yang berisi komponen seperti multi-head self-attention dan feedforward neural networks.

\subsubsection{Multi-Head Attention}

Komponen ini memungkinkan model untuk mempelajari hubungan antar token dalam urutan input dari berbagai representasi sub-ruang secara paralel.

\subsubsection{Positional Encoding}

Karena Transformer tidak memproses data secara sekuensial seperti RNN, maka diperlukan positional encoding untuk memberi informasi posisi urutan kepada model.

\subsection{Keunggulan Transformer dibandingkan RNN dan LSTM}

Transformer mengatasi keterbatasan RNN dan LSTM dalam hal ketergantungan jangka panjang, waktu pelatihan yang lambat, dan ketidakmampuan memanfaatkan paralelisme secara penuh pada GPU.
% \input{bab3}
% \setcounter{chapter}{99}  % Nomor bab tinggi agar tidak bentrok
\customchapter{}{DAFTAR PUSTAKA}
\renewcommand{\thesection}{}
\thispagestyle{plain}
\sloppy

\begin{enumerate}[label={[\arabic*]}, leftmargin=2.5em, itemsep=0.75em]

\item Goldberg, D. E. (1989). \textit{Genetic Algorithms in Search, Optimization, and Machine Learning}. Addison-Wesley.

\item Vaswani, A., Shazeer, N., Parmar, N., Uszkoreit, J., Jones, L., Gomez, A. N., ... \& Polosukhin, I. (2017). Attention is All You Need. \textit{Advances in Neural Information Processing Systems}, 30.

\item Yang, J. dan Xing, C. (2019). Data Source Selection Based on an Improved Greedy Genetic Algorithm. \textit{Symmetry}, 11(2), 273. Diakses dari: \url{https://doi.org/10.3390/sym11020273}

\item Goodfellow, I., Bengio, Y., \& Courville, A. (2016). \textit{Deep Learning}. MIT Press.

\item Suyanto. (2018). \textit{Machine Learning: Membangun Mesin Pintar dengan Python}. Informatika Bandung.

\item Sugiyono. (2013). \textit{Metode Penelitian Kuantitatif, Kualitatif, dan R\&D}. Alfabeta.

\end{enumerate}


% =========================
% DAFTAR PUSTAKA
% =========================
\clearpage
\renewcommand{\bibname}{DAFTAR PUSTAKA}
\addcontentsline{toc}{chapter}{DAFTAR PUSTAKA}
\bibliographystyle{apalike}  % Gaya mirip skripsi Indonesia
\bibliography{pustaka}



\end{document}
